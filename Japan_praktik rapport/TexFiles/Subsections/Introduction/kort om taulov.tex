Taulov er et af Danmarks største ostemejerier. I sommeren i 1998 begyndte byggeprojektet af det dengang \SI{}{30.000\metre\squared} og allerede i år 2000 stod mejeriet klart, og produktionen af de første officielle oste begyndt. 
Der har sidenhen været flere tilbygninger og mejeriet er vokset. I dag er mejeriet hele \SI{}{56.000\metre\squared} og har en produktions kapacitet på hele \SI{}{70.000\tonne} pro anno. Ifølge Arla selv, er Taulov måske verdens mest avancerede mejeri til fremstilling af ost, herunder bland andet Danbo, Havarti og Maribo. Der er omkring 140 medarbejdere på Taulov og ca. 70 tanksvognschauffører tilknyttet mejeriet. 
Arla har endvidere besluttet at lukke produktionen på Tistrup mejeri og flytte dennes produktion til Taulov, hvorfor det kan forventes at Taulov vil fortsætte sin hidtidige vækst på areal såvel som produktion af ost. 
Taulov ligger en dyd i at nedsætte energi forbug og dermed klima-aftryk. Målet er at benytte sig af 100\% grøn energi til år 2025, at nedsætte energiforbruget med 63\% til år 2030 og ikke mindst at være 100\% \ch{CO2} neutrale til år 2050. 

I denne rapport vil der blive lagt fokus på oste produktionen i Taulov fra A-Z. Dermed vil der blive beskrevet alle led fra køerne bliver malket og tankvognen fyldes, helt til osten er klar til ekspedition og distribuering til detailhandel. 

